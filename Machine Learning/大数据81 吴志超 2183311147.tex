\documentclass[12pt,a4paper]{article}
\usepackage{ctex}
\usepackage{mathtools}
\usepackage{amsmath,amssymb}
\usepackage{setspace}
\usepackage{geometry}
\usepackage{bm}
\usepackage{extarrows}
\geometry{left=3.0cm,right=3.0cm,top=2.0cm,bottom=2.0cm}

\title{机器学习第四次作业}
\author{大数据81 吴志超 2183311147}
\begin{document}
\begin{spacing}{1.3}
\section*{11.02作业}
\subsubsection*{1. 推导三硬币模型的 EM 算法中隐变量后验分布的计算公式以及参数更新公式}
\par \noindent (1)变量说明:
\par 观测变量X:一次实验观测到的结果,正面向上记为1,反面向上记为0;
$$X=(x_{1},\dots,x_{n} \in \{0,1\})$$ 
\par 隐变量Z:一次实验中,未观测到的硬币A的抛掷结果,正面向上记为1,反面向上记为0;
$$Z=(z_{1},\dots,z_{n}) \in \{0,1\} $$
\par 模型参数:$\theta=(\pi,p,q)$,分别代表硬币A、B、C抛掷得到正面向上的概率;
\\
\par 根据模型,在一次实验中:
$$P(x\ |\ \theta) = \sum\limits_{z} P(x,z\ |\ \theta) = \sum\limits_{z} P(z\ |\ \theta)·P(x\ |\ z,\theta) $$
\par 模型参数的极大似然估计为:
\begin{equation}
    \begin{split}
        L &= \prod\limits_{i=1}^{n} \ P(x_{i} \ | \ \theta) \\
        \log \ L &=\sum\limits_{i=1}^{n} \log P(x_{i} \ | \ \theta) \\
        &=\sum\limits_{i=1}^{n} \log \biggl(\pi p^{x_{i}}(1-p)^{1-x_{i}}+(1-\pi)q^{x_{i}}(1-q)^{1-x_{i}} \biggl) 
    \nonumber
    \end{split}
\end{equation}
\par 由于上式的log中出现求和形式,不便于求解,因此引入Jessen不等式:
$$E(f(x)) \ge f(E(x)) $$
\par 同时,假设隐变量的分布为Q(z),满足:
$$Q(z)\ge 0 \ , \ \sum\limits_{z} Q(z) =1$$
\par 因此,极大似然函数可以写作:
\begin{equation}
    \begin{split}
        LL = \log \ L &=\sum\limits_{i=1}^{n}\log \sum\limits_{z_{i}}Q(z_{i})
        \biggl(\dfrac{P(x_{i},z_{i}\ | \ \theta)}{Q(z_{i})} \biggr)  \\
        &\ge \sum\limits_{i=1}^{n}\sum\limits_{z_{i}}Q(z_{i}) 
        \log \biggl(\dfrac{P(x_{i},z_{i}\ | \ \theta)}{Q(z_{i})} \biggr) \\ 
        &=J(\theta,Q(z))  \nonumber
    \end{split}
\end{equation}
\par 这里,我们将$\biggl(\dfrac{P(x_{i},z_{i}\ | \ \theta)}{Q(z_{i})} \biggr)$看作是一个随机
变量,因此$\sum\limits_{z_{i}}Q(z_{i})\biggl(\dfrac{P(x_{i},z_{i}\ | \ \theta)}{Q(z_{i})} \biggr) $
可以看作是期望,根据Jessen不等式,上述不等式关系成立。为了取到极大似然函数的最大值,我们可以通过不断提高它的下界(使用EM算法)来实现。
\par 对于EM算法,在E步,我们需要实现:
$$Q^{t}(z) = \arg\max\limits_{Q(z)}\ J(\theta^{(t)},Q(z)) $$
\par 为了让下界取到最大值,我们希望上述不等式关系的能成为等式关系,
在Jessen不等式中,不等式当且仅当变量X为常量时取到等号,因此,我们令:
$$ \dfrac{P(x_{i},z_{i}\ | \ \theta)}{Q(z_{i})} = const = C$$
$$Q(z_{i})=\dfrac{P(x_{i},z_{i}\ | \ \theta)}{C} $$
\par 根据隐变量分布的性质,有:
$$\sum\limits_{z_{i}} Q(z_{i}) = \sum\limits_{z_{i}} \dfrac{P(x_{i},z_{i}\ | \ \theta)}{C} = 1$$
$$C = \sum\limits_{z_{i}} P(x_{i},z_{i}\ | \ \theta) $$ 
\par 因此,$Q(z_{i})$可写作:
\begin{equation}
    \begin{split}
        Q(z_{i}) & = \dfrac{P(x_{i},z_{i}\ | \ \theta^{(t)})}{\sum\limits_{z_{i}}P(x_{i},z_{i}\ | \ \theta^{(t)})} 
         = \dfrac{P(x_{i},z_{i}\ | \ \theta^{(t)})}{P(x_{i} \ | \ \theta^{(t)})} \\
        &=P(z_{i} \ | \ x_{i},\theta^{(t)}) = \dfrac{P(x_{i}\ | \ z_{i},\theta^{(t)})·P(z_{i}\ | \ \theta^{(t)})}{P(x_{i} \ | \ \theta^{(t)})} \\
        & = \dfrac{  \biggl( \pi p^{x_{i}}(1-p)^{1-x_{i}} \biggr)^{z_{i}} ·
        \biggl( (1-\pi)q^{x_{i}}(1-q)^{1-x_{i}} \biggr)^{1-z_{i}}  }
        {\pi p^{x_{i}}(1-p)^{1-x_{i}}+(1-\pi)q^{x_{i}}(1-q)^{1-x_{i}}} \nonumber 
    \end{split}
\end{equation}
\par 在M步,$Q^{(t)}(z_{i})$为常数,因此有:
\begin{equation}
    \begin{split}
        \theta^{(t+1)} &= \arg\max\limits_{\theta} \ J(\theta,Q^{t}(z)) \\
        & =  \arg\max\limits_{\theta} \ \sum\limits_{i=1}^{n} \sum\limits_{z_{i}}   Q^{(t)}(z_{i}) \log \ P(x_{i},z_{i} \ | \theta) \\
        & =  \arg\max\limits_{\theta} \ \sum\limits_{i=1}^{n} \sum\limits_{z_{i}}   Q^{(t)}(z_{i}) \log \ P(x_{i} \ | z_{i},\theta)·P(z_{i}\ | \ \theta) \\
        & =  \arg\max\limits_{\pi,p,q} \ \sum\limits_{i=1}^{n} \sum\limits_{z_{i}} Q^{(t)}(z_{i})    
        *\biggl(z_{i} \log (\pi p^{x_{i}}(1-p)^{1-x_{i}}) + 
        (1-z_{i})\log ((1-\pi)q^{x_{i}}(1-q)^{1-x_{i}}) \biggr) \\
        &=
        \arg\max\limits_{\pi,p,q} \ \sum\limits_{i=1}^{n} \biggl(
        \gamma_{i,0} \log \pi p^{x_{i}}(1-p)^{1-x_{i}}  + \gamma_{i,1} \log (1-\pi)q^{x_{i}}(1-q)^{1-x_{i}}   \biggr) 
        \nonumber
    \end{split}
\end{equation}
\par 其中,$\gamma_{i,0}=Q^{(t)}(z_{i}=0) , \gamma_{i,1}=Q^{(t)}(z_{i}=1)$
\par 求导得:
$$\dfrac{\partial J}{\partial \pi^{t}} = -\sum\limits_{i=1}^{n}\gamma_{i,0}\dfrac{1}{1-\pi^{t}} 
+ \sum\limits_{i=1}^{n}\gamma_{i,1}\dfrac{1}{\pi_{t}}$$
$$\dfrac{\partial J}{\partial q^{t}} =
\sum\limits_{i=1}^{n}\gamma_{i,0}\dfrac{x_{i}}{q^{t}}-\sum\limits_{i=1}^{n}\gamma_{i,0}\dfrac{1-x_{i}}{1-q^{t}} $$
$$\dfrac{\partial J}{\partial p^{t}} =
\sum\limits_{i=1}^{n}\gamma_{i,1} \dfrac{x_{i}}{p^{t}} - \sum\limits_{i=1}^{n}\gamma_{i,1}\dfrac{1-x_{i}}{1-p^{t}} $$
\par 推导如下:
\par 对于$\pi^{t}$
$$\sum\limits_{i=1}^{n}\gamma_{i,0}\dfrac{1}{1-\pi^{t}} =\sum\limits_{i=1}^{n}\gamma_{i,1}\dfrac{1}{\pi_{t}}$$
$$\sum\limits_{i=1}^{n}\gamma_{i,0} = \sum\limits_{i=1}^{n}\gamma_{i,1}(\dfrac{1}{\pi^{t}}-1) $$
$$N = \sum\limits_{i=1}^{n}(\gamma_{i,0}+\gamma_{i,1}) = \sum\limits_{i=1}^{n}\gamma_{i,1}\dfrac{1}{\pi^{t}} $$
$$\pi^{t} = \dfrac{1}{n} \sum\limits_{i=1}^{N}\gamma_{i,1}$$
\par 对于$q^{t}$
$$\sum\limits_{i=1}^{n}\gamma_{i,0} \dfrac{x_{i}}{q^{t}} = \sum\limits_{i=1}^{n}\gamma_{i,0} \dfrac{1-x_{i}}{1-q^{t}} $$
$$(\dfrac{1}{q^{t}}-1) = \dfrac{\sum\limits_{i=1}^{n}\gamma_{i,0}x_{i}}{\sum\limits_{i=1}^{n}\gamma_{i,0}(1-x_{i})} $$
$$q^{t} = \dfrac{1}{1+\dfrac{\sum\limits_{i=1}^{n}\gamma_{i,0}x_{i}}{\sum\limits_{i=1}^{n}\gamma_{i,0}(1-x_{i})}}
= \dfrac{\sum\limits_{i=1}^{n}\gamma_{i,0}(1-x_{i})} {\sum\limits_{i=1}^{n}\gamma_{i,0}(1-x_{i})+\sum\limits_{i=1}^{n}\gamma_{i,0}x_{i}}
= \dfrac{\sum\limits_{i=1}^{n}\gamma_{i,0}(1-x_{i})}{\sum\limits_{i=1}^{n}\gamma_{i,0}}$$
\par 对于$p^{t}$,同$q^{t}$推导过程,可得
$$p^{t} = \dfrac{1}{1+\dfrac{\sum\limits_{i=1}^{n}\gamma_{i,1}x_{i}}{\sum\limits_{i=1}^{n}\gamma_{i,1}(1-x_{i})}}
= \dfrac{\sum\limits_{i=1}^{n}\gamma_{i,1}(1-x_{i})} {\sum\limits_{i=1}^{n}\gamma_{i,1}(1-x_{i})+\sum\limits_{i=1}^{n}\gamma_{i,1}x_{i}}
= \dfrac{\sum\limits_{i=1}^{n}\gamma_{i,1}(1-x_{i})}{\sum\limits_{i=1}^{n}\gamma_{i,1}}$$
\subsubsection*{2. 推导高斯混合模型的 EM 算法中 M 步的参数更新公式}
\par 不难写出,高斯混合模型X的边缘分布为:
$$P(x) = \sum\limits_{z} P(x,z) = \sum\limits_{z} P(x\ | \ z)·P(z) = \sum\limits_{k=1}^{K}
\alpha_{k} P(x \ | \ \mu_{k}, \epsilon_{k}) $$
\par 其中$\epsilon_{k}$代表第k个高斯的协方差矩阵,$\mu_{k}$代表第k个高斯的均值,$\alpha_{k}$
代表样本x由第k个高斯生成的概率。
\par 记模型的参数为$\theta=\{\alpha_{k},\mu_{k},\epsilon_{k} \}_{k=1}^{K}$,那么参数的极大似然
函数可写作:$$L(\theta \ | \ x) = \prod\limits_{i=1}^{n} P(x_{i}\ | \ \theta) $$
\par 对数似然函数可写作:
$$LL(\theta \ | \ x) = \sum\limits_{i=1}^{n} \log \ \sum\limits_{k=1}^{K} \alpha_{k}P(x \ | \ \mu_{k}, \epsilon_{k}) $$
\subsubsection*{E步:}
\par 与1中同理,我们可以引入一个隐变量$z_{i}$的分布$Q(z_{i})$,通过推导能够得到EM算法的E步得到的
$Q(z_{i})$为其后验分布:
$$Q(z_{i})= P(z_{i}=k\ | x_{i})=\dfrac{P(x_{i}\ | z_{i}=k)·P(z_{i}=k)}{P(x_{i})}
=\dfrac{\alpha_{k} P(x_{i} \ | \ \mu_{k}, \epsilon_{k})}{\sum\limits_{k=1}^{K}\alpha_{k} P(x_{i} \ | \ \mu_{k}, \epsilon_{k}) }$$
\par 将$Q(z_{i})$记作$\gamma_{i,k}$,在M中为一常数。
\subsubsection*{M步:}
\begin{equation}
    \begin{split}
        \theta^{+} 
        &=\arg\max\limits_{\theta}  \sum\limits_{i=1}^{n} \sum\limits_{k=1}^{K} Q(z_{i})\log \dfrac{\alpha_{k} P(x_{i} \ | \ \mu_{k}, \epsilon_{k})}{Q(z_{i})} \\
        &=\arg\max\limits_{\theta}  \sum\limits_{i=1}^{n} \sum\limits_{k=1}^{K} \gamma_{i,k} \log \alpha_{k} P(x_{i} \ | \ \mu_{k}, \epsilon_{k}) \\
        &=\arg\max\limits_{\theta}  \sum\limits_{i=1}^{n} \sum\limits_{k=1}^{K} \gamma_{i,k} \biggl(\log P(x_{i} \ | \ \mu_{k}, \epsilon_{k}) + \log \alpha_{k} \biggr)
        \\ &=\arg\max\limits_{\theta}  \sum\limits_{i=1}^{n} \sum\limits_{k=1}^{K} \gamma_{i,k} \biggl( -\dfrac{1}{2}\log |\epsilon_{k}|-\dfrac{1}{2} (x_{i}-\mu_{k})^{T}\epsilon_{k}^{-1}(x_{i}-\mu_{k}) + \log \alpha_{k}       \biggr)
        \\ &=\arg\min\limits_{\theta}  \sum\limits_{i=1}^{n} \sum\limits_{k=1}^{K} \gamma_{i,k} \biggl( \dfrac{1}{2}\log |\epsilon_{k}|+\dfrac{1}{2} (x_{i}-\mu_{k})^{T}\epsilon_{k}^{-1}(x_{i}-\mu_{k}) - \log \alpha_{k}       \biggr)
        \\ &=G(\alpha_{k},\mu_{k},\epsilon_{k})
        \nonumber
    \end{split}
\end{equation}
1. 令$  \dfrac{\partial G(\alpha_{k},\mu_{k},\epsilon_{k})}{\partial \mu_{k}} 
=\sum\limits_{i}^{n} \gamma_{i,k} \biggl( -(x_{i}-\mu_{k})^{T}\epsilon_{k}^{-1} \biggr) = 0   $,且$\epsilon_{k}^{-1} \not = \bf{O}$,因此有:
$$\sum\limits_{i}^{n} \gamma_{i,k} \biggl( -(x_{i}-\mu_{k})^{T} \biggr) = 0  \Longleftrightarrow 
\sum\limits_{i}^{n} \gamma_{i,k}\ x_{i} = \sum\limits_{i}^{n} \gamma_{i,k}\ \mu_{k}$$
\par 所以有:$\mu_{k} = \dfrac{\sum\limits_{i}^{n} (\gamma_{i,k}\ x_{i})}{\sum\limits_{i}^{n} \gamma_{i,k}} $ 
\\2. 令$  \dfrac{\partial G(\alpha_{k},\mu_{k},\epsilon_{k})}{\partial \epsilon_{k}} 
=\sum\limits_{i=1}^{n} \gamma_{i,k} \biggl( \dfrac{1}{2|\epsilon_{k}|}|\epsilon_{k}|\epsilon_{k}^{-1} - \dfrac{1}{2}(x_{i}-\mu_{k})(x_{i}-\mu_{k})^{T}\epsilon_{k}^{-1}I\epsilon_{k}^{-1} \biggr) = 0 $,有:
$$\sum\limits_{i=1}^{n} \gamma_{i,k}\ \epsilon_{k}^{-1}\epsilon_{k}\epsilon_{k} = \sum\limits_{i=1}^{n} \gamma_{i,k} (x_{i}-\mu_{k})(x_{i}-\mu_{k})^{T}\epsilon_{k}^{-1}I\epsilon_{k}^{-1}\epsilon_{k}\epsilon_{k} $$
\par 化简后得:
$$\sum\limits_{i=1}^{n} \gamma_{i,k}\ \epsilon_{k} = \sum\limits_{i=1}^{n} \gamma_{i,k}\ (x_{i}-\mu_{k})(x_{i}-\mu_{k})^{T}
\Longleftrightarrow \epsilon_{k} = \dfrac{\sum\limits_{i=1}^{n} \gamma_{i,k}\ (x_{i}-\mu_{k})(x_{i}-\mu_{k})^{T}}{\sum\limits_{i=1}^{n} \gamma_{i,k}}$$
\\3. 在对$\alpha_{k}$进行求导时不能直接进行,因为这里有一个隐形的约束条件:
$$\sum\limits_{k=1}^{K} \alpha_{k}=1 $$
\par 这个问题相当于求一个有约束的优化问题,因此将其转换成拉格朗日函数的形式:
$$L(\alpha,\lambda) = \sum\limits_{i=1}^{n}\sum\limits_{k=1}^{K} \gamma_{i,k} \log \alpha_{k} + \lambda(\sum\limits_{k=1}^{K} \alpha_{k} -1) $$
\par 求导后令其为0,得:
$$\sum\limits_{i=1}^{n} \gamma_{i,k} \dfrac{1}{\alpha_{k}} + \lambda = 0
\quad , \quad \sum\limits_{i=1}^{n} \gamma_{i,k}= -\lambda \alpha_{k}$$
\par 对每一个$\alpha_{k}$,上式成立,因此有:
$$\sum\limits_{k=1}^{K}\sum\limits_{i=1}^{n} \gamma_{i,k} =- \sum\limits_{k=1}^{K} \lambda \alpha_{k} = -\lambda = \sum\limits_{i=1}^{n} \gamma_{i,k} \dfrac{1}{\alpha_{k}}$$
\par 这里的$\gamma_{i,k}$实际上式关于隐变量z的后验分布,因此有:
$$ \sum\limits_{k=1}^{K} \gamma_{i,k} = 1 \longrightarrow \sum\limits_{i=1}^{n}\sum\limits_{k=1}^{K} \gamma_{i,k} = N$$
\par 最终得到$\alpha_{k}$的更新公式为:
$$\alpha_{k} = \dfrac{\sum\limits_{i=1}^{n} \gamma_{i,k}}{N} $$
\par 综上所述,M步的迭代更新公式为:
$$\mu_{k} = \dfrac{\sum\limits_{i}^{n} (\gamma_{i,k}\ x_{i})}{\sum\limits_{i}^{n} \gamma_{i,k}}
\ , \ \epsilon_{k}= \dfrac{\sum\limits_{i=1}^{n} \gamma_{i,k}\ (x_{i}-\mu_{k})(x_{i}-\mu_{k})^{T}}{\sum\limits_{i=1}^{n} \gamma_{i,k}}
\ , \  \alpha_{k} = \dfrac{\sum\limits_{i=1}^{n}\gamma_{i,k}}{N} $$

\end{spacing}
\end{document}